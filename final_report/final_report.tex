\documentclass[11pt]{article}

\usepackage[letterpaper]{geometry}
\usepackage{amsmath}
\usepackage{amsthm}
\usepackage{amssymb}
\usepackage[utf8]{inputenc}
\usepackage{url}
\newcommand{\reals}{{\mathbb R}}

\title{Final report\\CS6350 Machine Learning}
\author{Montgomery Carter \and Shaobo He}


\begin{document}
\maketitle

\section{Introduction}
The ability to predict a cuisine type of a recipe based on its
ingredients has some interesting real-world implications.  A retailer
can better target advertisements to its customers based on their food
preferences and buying habits.  For example, if a shopping cart
contains ingredients indicative of a certain cuisine, the retailer can
suggest other products that might be appealing to the consumer.

This is the problem that we address in our project.  The idea
originated from a competition on kaggle.com\cite{kaggle-link}. The
competition on  kaggle.com provides a list of recipes.  For each
recipe, the ingredients are given, as well as a cuisine type which
classifies each recipe.

We found this problem interesting in three ways. First, as mentioned,
the problem has real-world applications and might be interesting to
retailers, social media sites, and other organizations that aggregate
data about users and consumers.  Second, data given to us in this
project comes from real world, which provides us with an opportunity
to learn tricks to deal with unshaped data such as high
dimensionality, noise, and unseen labels.  Finally, we found this 
project to be interesting because it is a multi-class classification
problem with a relatively high number of classes.  Because we didn't
get much time to address multi-class classification in class, it has
been nice to get some experience with such problems.

Because this was our first real-world application of machine learning,
we didn't have much experience to guide our selection of a machine
learning algorithm.  As a result, our project became an exploration of
various machine learning algorithms.  We applied multiple machine
learning algorithms that we learned from class by leveraging
scikit-learn python package. Though we lacked real-world application
experience, we used theory learned in class to help guide our search
through the dozens of algorithms available in the scikit-learn
package.

In this paper we describe our investigation into the classification of
the cuisine of a recipe based on its ingredients.

\section{Problem Description and Preliminary Analysis}
As mentioned, we focus on classifying the cuisine of a recipe given
its ingredients.  we found pre-compiled recipe data on kaggle.com.
This dataset includes two json files, one for training data, and one
for test data.  The training data is a list of recipes.  For each
recipe, a list of ingredients and a cuisine are given.  The test data
is a list of recipes, which only provides ingredients for each
recipe. 

A preliminary analysis of the data reveals some basic statistics about
the dataset.  The original training data includes 39,744 recipes taken
from 20 different cuisines, which use a total of 6,714 different
ingredients.  Of these 39,744 recipes, 7,838 are Italian, 6,438 are
Mexican, 4,320 are Southern US, 3,003 are Indian, 2,673 are Chinese,
and 2,646 are French.  Each of the remaining 14 cuisines contribute
fewer than 1,500 recipes each.

For a majority of cuisines, the most commonly used ingredients are
common across all cuisines.  For example, the most commonly used
ingredient of all cuisines is salt or a salt replacement like soy
sauce or fish sauce.  Although many of the top ingredients are common
across all cuisines, there are ingredients that we refer to as
indicator ingredients.  Such ingredients are only seen in a small
number of cuisines.  For example, garam marsala is \emph{only} used in
Indian food, where it is found in approximately 1/3 of all Indian
recipes.  Another example is fish sauce, which is only found in Asian
cuisines (Chinese, Korean, Filipino, etc.) Such ingredients
undoubtedly assist the prediction process as they reduce 
the number of cuisines possible for a recipe using such ingredients.  

Another observation to be noted for our problem is that a single set of
ingredients can map to multiple cuisines.  For example, French
baguettes and Italian breadsticks may call for the same ingredients in
different quantities.


\section{Methodology}
We chose to implement our project Python due to its ability to rapidly
prototype solutions, our familiarity with the language, and the
availability of scikit-learn, an easy to use, well supported machine
learning package.  

We first perform preprocessing on the data.  Because we the test set
provided by Kaggle does not include labels, we partition 90\% the
provided training data as our training set, and 10\% as our test set.
This provides us with the ability to measure the performance of our
final solution.  Two sets are then generated -- one containing the
ingredients and one containing the cuisine labels in use across the
training set.  These lists are then sorted.  The index of each
ingredient in the sorted set of ingredients becomes the feature
dimension number for that ingredient, and the index of each cuisine in
the sorted set of cuisines becomes the label number for each cuisine
class.  A vector is then created for each recipe in the training and
test sets.  For each vector, the features get set to 1.0 if the
ingredient is present in the recipe, and the cuisine number becomes
the last element in the vector.

As this is our first application of machine learning to a real-world
problem, we didn't have much experience to guide our selection of an
algorithm.  Because of this, our project became 
Because 


\section{Results}


\section{Analysis}


\section{Future Work}


\section{Conclusion}


\bibliography{refs}
\bibliographystyle{plain}
\end{document}
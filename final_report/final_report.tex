\documentclass[11pt]{article}

\usepackage[letterpaper]{geometry}
\usepackage{amsmath}
\usepackage{amsthm}
\usepackage{amssymb}
\usepackage[utf8]{inputenc}
\usepackage{url}
\usepackage{caption}
\newcommand{\reals}{{\mathbb R}}

\title{Project Report\\CS6350 Machine Learning}
\author{Montgomery Carter \and Shaobo He}
\date{\vspace{-5ex}}

\begin{document}
\maketitle

\section{Introduction}
The ability to predict a cuisine type of a recipe based on its
ingredients has some interesting real-world implications.  A retailer
can better target advertisements to its customers based on their food
preferences and buying habits.  For example, if a shopping cart
contains ingredients indicative of a certain cuisine, the retailer can
suggest other products that might be appealing to the consumer.

This is the problem that we address in our project.  The idea
originated from a competition on kaggle.com\cite{kaggle-link}. The
competition on  kaggle.com provides a list of recipes.  For each
recipe, the ingredients are given, as well as a cuisine type which
classifies each recipe.

We found this problem interesting in three ways. First, as mentioned,
the problem has real-world applications and might be interesting to
retailers, social media sites, and other organizations that aggregate
data about users and consumers.  Second, data given to us in this
project comes from the real world and provides us with an opportunity
to learn tricks to deal with high-dimensional, noisy, raw data.
Finally, we found this project to be interesting because it is a
multi-class classification problem with a relatively high number of
classes.  Because we didn't get much time to address multi-class
classification in class, it has been nice to get some experience with
such problems. 

Because this was our first real-world application of machine learning,
we didn't have much experience to guide our selection of a machine
learning algorithm.  As a result, our project became an exploration of
various machine learning algorithms.  We applied multiple machine
learning algorithms that we learned from class by leveraging
scikit-learn python package. Though we lacked real-world application
experience, we used theory learned in class to help guide our search
through the dozens of algorithms available in the scikit-learn\cite{scikit-learn}
package.

In this paper we describe our investigation into the classification of
the cuisine of a recipe based on its ingredients.

\section{Problem Description and Preliminary Analysis}
As mentioned, we focus on classifying the cuisine of a recipe given
its ingredients.  We found pre-compiled recipe data on kaggle.com.
This dataset includes training data in a json file format.  The
training data is a list of recipes.  For each recipe, a list of
ingredients and a cuisine are given.  

A preliminary analysis of the data reveals some basic statistics about
the dataset.  The original training data includes 39,744 recipes taken
from 20 different cuisines, which use a total of 6,714 different
ingredients.  Of these 39,744 recipes, 7,838 are Italian, 6,438 are
Mexican, 4,320 are Southern US, 3,003 are Indian, 2,673 are Chinese,
and 2,646 are French.  Each of the remaining 14 cuisines contribute
fewer than 1,500 recipes each.

For a majority of cuisines, the most frequently used ingredients are
common across cuisines.  For example, the most commonly used
ingredient of all cuisines is salt or a salt replacement like soy
sauce or fish sauce.  Although several of the top ingredients are
common across all cuisines, there are ingredients that we refer to as
indicator ingredients.  Such ingredients are only seen in a small
number of cuisines.  For example, garam marsala is \emph{only} used in
Indian food, where it is found in approximately 1/3 of all Indian
recipes.  Another example is fish sauce, which is only found in Asian
cuisines (Chinese, Korean, Filipino, etc.) Such ingredients
undoubtedly assist the prediction process as they reduce 
the number of cuisines possible for a recipe using such ingredients.  

Another observation to be noted for our problem is that a single set of
ingredients can map to multiple cuisines.  For example, French
baguettes and Italian breadsticks may call for the same ingredients in
different quantities.  This necessarily places a limit on the accuracy
we can hope to obtain in classifying recipes based on their
ingredients alone.


\section{Methodology}
We chose to implement our project Python due to its ability to rapidly
prototype solutions, our familiarity with the language, and the
availability of scikit-learn, an easy to use, well supported machine
learning package.  

We first perform preprocessing on the data.  Because there was no
useful test set provided on kaggle.com, we partition 90\% of the
provided training data as our ``training set'', and 10\% as our ``test
set''. This provides us with the ability to measure the performance of
our final solution.  Two sets are then generated -- one containing the
ingredients and one containing the cuisine labels in use across the
training set.  These lists are then sorted.  The index of each
ingredient in the sorted set of ingredients becomes the feature 
dimension number for that ingredient, and the index of each cuisine in
the sorted set of cuisines becomes the label number for each cuisine
class.  A vector is then created for each recipe in the training and
test sets.  For each vector, the features get set to 1.0 if the
ingredient is present in the recipe, and the cuisine number becomes
the last element in the vector.  The vectors are then converted into
compressed row storage (CSR) format (this turned out to significantly
improve the performance of both reading and preprocessing the data, as 
well as the performance of the fitting and prediction).

As this is our first application of machine learning to a real-world
problem, we didn't have much experience to guide our selection of an
algorithm.  We began our investigation with a support vector machine
classifier.  After experimenting with several different SVM
configurations, we then proceeded to try two other classes of
learners: Naive Bayes and Decision Trees.  Initial results from these
categories of learners were encouraging.  However, seeing these
initial results, it became clear that we would likely see improvement
by using some ensemble methods.  Each of the algorithms used
automatically uses the concept of epochs (except K-Nearest Neighbors),
and we used the default stopping criteria to determine when to
terminate learning.  

For each of the algorithms used, 10-fold cross validation was used on
the training set.  For all algorithms taking hyper-parameters, a wide
array of values was tried for each hyper-parameter.  The
hyper-parameter combination yielding the best accuracy for each
algorithm tried is provided in the results section.  Our cross
validation framework utilizes a thread pool, allowing us to perform
cross-validation in a highly parallelized fashion.

After performing cross-validation for each algorithm, the best
hyper-parameter combination was then used to retrain the classifier
using the full training set, and the resulting classifier was used
to predict labels of the test set.  The results of these experiments
are seen below in the results section. Our implementation can be found
on Github (\url{https://github.com/MontyCarter/CuisineClassifier}).

We ran machine learning algorithms on Emulab d820\cite{emulab-wiki}
machines with four 2.2 GHz 64-bit 8-Core Intel E5-4620 processors and
128GB DDR3 memory. We utilized all the CPUs when doing
multi-processing cross validation. 

\section{Results}
We ran machine learning algorithms on Emulab d820 machines with four 2.2 GHz 64-bit 8-Core Intel E5-4620 processors and 128GB DDR3 memory. We utilized all the CPUs when doing multi-processing cross validation.

Experimental results are shown in the following table. Machine learning algorithms are grouped in the table and groups are separated with an extra line. Accuracy is the average prediction accuracy of ten fold cross validation. Time is the average time taken to complete one fold. We give averaged time instead of total time because the average time is approximate to the runtime for the whole training set. We only show part of the best hyper parameters which is different to the default settings of each machine learning algorithm. Moreover, NA in the optimal hyper parameter column for a machine learning algorithm indicates that there are few hyper parameters that we can choose or we believe cross validation would not improve accuracy significantly for that particular algorithm.

\begin{center}
	\captionof{table}{Experimental results}
	\begin{tabular}{c|c|c|c}
    \hline
	     Algorithm   & Accuracy & Time & Optimal hyper parameter\\
         \hline
         SVM SVC (rbf kernel)   & 77.83  & 839.8 & C: 10.0, gamma: 0.1\\
         \hline
         SVM SVC (poly kernel)  & 77.35  & 430.1 & C: 1000.0, coef0: 1.0, degree: 2\\
         \hline
         SVM SVC (linear kernel)   &  76.13  & 223.0 & C: 1.0\\
         \hline
         SVM LinearSVC    &  77.21   &362.9  & C: 1.0\\
         \hline
         \hline
         Gaussian Naive Bayes & 35.23 & 0.23 & NA \\
         \hline
         Bernoulli Naive Bayes &   70.71    & 0.28  & NA \\
         \hline
         Multinomial Naive Bayes & 73.62 	& 0.23  & NA \\
         \hline
         \hline
         Decision Tree &  61.23 & 10.1 & NA \\
         \hline
         \hline
         K nearest neighbors & 54.11 & 11.26 & k: 15, weights:  distance\\
         \hline
         \hline
         Random Forest & 72.10 & 156.59 & n\_estimators: 100\\
         \hline
         Extra Trees & 73.44 & 356.2 &
         min\_samples\_split: 4,
         n\_estimators: 200 \\
         \hline
         AdaBoost (decision tree) & 55.66 & 126.4 & max\_depth: 1, learning\_rate : 1.0, n\_estimators: 100\\
         \hline
         AdaBoost (SVC) & 19.71 & 12057.7 & learning\_rate: 1.0,
         n\_estimators: 10\\
         \hline
        
          \end{tabular}
\end{center}


%\section{Analysis}
\section{Analysis}
The general lesson that we learned from this project is machine learning is not easy, specially for multi-class classification. Looking at the results shown in table 1, it is clear that we reached a limit on accuracy (around 80\%). Our guess for such limit is that multiple cuisines may share the same set of ingredients as mentioned in the previous section. For example, a large proportion of ingredients from Asian cuisines is the same. A direct implication of ingredient similarity among cuisines is that data set is not linearly separable. Even with feature transformation implemented by SVM with non-linear kernels, we were not be able to improve the performance of classifiers. Also some recipes only contain common ingredients such as salt, water, and butter, which makes prediction even harder.

Moreover, we also learned that complexity of machine learning algorithms matters. One can see that runtime as well as accuracy of SVM algorithms is proportional to the complexity of their kernels. We would expect longer runtime if we increase the degree of polynomial kernels. Therefore, a trade-off between runtime (probably accuracy) and algorithm complexity can be explore. We also noticed resource limit is also an important factor to consider when we choose machine learning algorithms. We encountered out of memory exceptions when running K-nearest neighbor algorithm even used sparse matrix that reduces the size of input data significantly compared to dense array. As a result, we had to reduce the number of processes and thus were not be able to fully utilize the computing capacity of powerful Emulab machines.

For the rest of the section, we give our analysis on machine learning algorithms that we tried in this project with a focus on why some of them works while others do not.

\paragraph{SVM} SVM gives us best accuracy albeit its long runtime. We think indicator ingredients probably contribute to it because intuitively they tend to increase the margin and make dataset more linearly separable. Note that polynomial and Gaussian kernel does not improve the accuracy a lot, which means that boundaries among cuisines is approximately linear and we could improve prediction accuracy after pruning the data. 

\paragraph{Probabilistic learning} After some analysis on statistics that we gathered from the data set, we found it straight forward to explain why the performance of probabilistic learning is comparable to the best we have. Recall that the prediction rule for naive Bayes classifier is $argmax_y \Pr(Y)\prod_{i} \Pr(x_i|Y)$, which means that if a cuisine has some indicator ingredients (the probability that it belongs to a certain cuisine is high), then it is highly likely to be predicted correctly. It corresponds to our observation that the top five ingredients of 20 cuisines are mostly different.
 
\paragraph{Ensemble methods} Ensemble methods with decision trees as weak learners are proved to be effective according to the result. We think the reason why it works is explained in class while we cannot give a reasonable explanation on why the accuracy of AdaBoost with SVC is low.

\paragraph{Those that do not work well} Decision trees are known to overfit the training set. We tried to mitigate overfitting by setting up a depth limit while the accuracy was not improved. For K nearest neighbor, we found it confusing because theoretically it is very expressive and thus should be able to provide high accuracy. Our assumption is that we might not choose the best distance measurement. We only used Euclidean distance instead of trying other distance measurement due to lack of efficient implementation of these algorithms in sci-kit library.


\section{Future Work}
Given extra time, we would like to improve our work in following
directions. 
\begin{enumerate}
	\item We would like to prune the dataset by merging
          ingredients which have different names while are essentially
          the same.   
	\item It would be good for us to analyze mislabeled data so
          that we may have useful observation to facilitate learning.
	\item Although we have used several machine learning
          algorithms in this project, we would like to try more
          parameter combinations of AdaBoost algorithm, particularly
          with the SVM kernel (surely, we should be able to get better
          than chance!)
\end{enumerate}

\section{Conclusion}
Although we saw that multi-class classification can be difficult and
may lead to lower prediction accuracy, I think that we obtained a
reasonably high rate of around 80\%.  

We believe there may be an limit to the accuracy that is possible with
even the best algorithm, due to the reasons described in the beginning
of the analysis section. And we proposed a method to push the limit in
the future work section. 
%the fact that different cuisines can
%be indicated by the same set of ingredients.

%We saw that ensemble methods can definitely help to boost performance
%of single algorithms.

Overall we were pleased with the results of our implementation.  It
was a good experience, and it gave us a great opportunity to try
various learning algorithms on real world data.

\bibliography{refs}
\bibliographystyle{plain}
\end{document}
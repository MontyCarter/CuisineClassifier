\documentclass[11pt]{article}

\usepackage[letterpaper]{geometry}
\usepackage{amsmath}
\usepackage{amsthm}
\usepackage{amssymb}
\usepackage[utf8]{inputenc}
\usepackage{url}
\newcommand{\reals}{{\mathbb R}}

\title{Final report\\CS6350 Machine Learning}
\author{Montgomery Carter \and Shaobo He}


\begin{document}
\maketitle

\section{Introduction}
The problem that we addressed in our project is to predict the cuisine type of a
recipe based on the recipe's ingredients. It originates from a competition on kaggle.com\cite{kaggle-link}. The setting of this problem is that we are given a training set of different cuisines and each cuisine is associated with a list of ingredients used to make it.

We found it interesting in two ways. First, we can possibly learn the relation between recipe ingredients and ethnicity via this project. For example, a local grocery store may advertise more precisely to customers after an analysis of the category of their purchases based on the learned model. Second, data given to us in this project comes from real world, which can help us to learn the tricks to deal with unshaped data such as high dimensionality, noise, and unseen labels.

We applied multiple machine learning algorithms that we learned from class by leveraging sci-kit learn python package. We also used machine learning algorithms whose underlying ideas are covered in class though their details are not mentioned. For the following sections, we will describe our hypothesis of this problem and how we tried to learn the hypothesis using corresponding machine learning algorithms.

\section{Results}
We ran machine learning algorithms on Emulab d820 machines with four 2.2 GHz 64-bit 8-Core Intel E5-4620 processors and 128GB DDR3 memory. We utilized all the CPUs when doing multi-processing cross validation.

Experimental results are shown in the following table. Machine learning algorithms are grouped in the table and groups are separated with an extra line. Accuracy is the average prediction accuracy of ten fold cross validation. Time is the average time taken to complete one fold. We give averaged time instead of total time because the average time is approximate to the runtime for the whole training set. We only show part of the best hyper parameters which is different to the default settings of each machine learning algorithm. Moreover, NA in the optimal hyper parameter column for a machine learning algorithm indicates that there are few hyper parameters that we can choose or we believe cross validation would not improve accuracy significantly for that particular algorithm.

\begin{center}
	\captionof{table}{Experimental results}
	\begin{tabular}{c|c|c|c}
    \hline
	     Algorithm   & Accuracy & Time & Optimal hyper parameter\\
         \hline
         SVM SVC (rbf kernel)   & 77.83  & 839.8 & C: 10.0, gamma: 0.1\\
         \hline
         SVM SVC (poly kernel)  & 77.35  & 430.1 & C: 1000.0, coef0: 1.0, degree: 2\\
         \hline
         SVM SVC (linear kernel)   &  76.13  & 223.0 & C: 1.0\\
         \hline
         SVM LinearSVC    &  77.21   &362.9  & C: 1.0\\
         \hline
         \hline
         Gaussian Naive Bayes & 35.23 & 0.23 & NA \\
         \hline
         Bernoulli Naive Bayes &   70.71    & 0.28  & NA \\
         \hline
         Multinomial Naive Bayes & 73.62 	& 0.23  & NA \\
         \hline
         \hline
         Decision Tree &  61.23 & 10.1 & NA \\
         \hline
         \hline
         K nearest neighbors & 54.11 & 11.26 & k: 15, weights:  distance\\
         \hline
         \hline
         Random Forest & 72.10 & 156.59 & n\_estimators: 100\\
         \hline
         Extra Trees & 73.44 & 356.2 &
         min\_samples\_split: 4,
         n\_estimators: 200 \\
         \hline
         AdaBoost (decision tree) & 55.66 & 126.4 & max\_depth: 1, learning\_rate : 1.0, n\_estimators: 100\\
         \hline
         AdaBoost (SVC) & 19.71 & 12057.7 & learning\_rate: 1.0,
         n\_estimators: 10\\
         \hline
        
          \end{tabular}
\end{center}
\bibliography{refs}
\bibliographystyle{plain}
\end{document}